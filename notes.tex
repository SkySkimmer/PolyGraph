\documentclass[12pt]{article}
\usepackage[english]{babel}
\usepackage[utf8]{inputenc}
\usepackage[T1]{fontenc}
\usepackage{amsfonts}
\usepackage{amsmath}
\usepackage{amssymb}
\usepackage{listings}
\usepackage{color}
\usepackage{units}
\usepackage{mathtools}
\usepackage[bookmarks]{hyperref}
\usepackage[super]{nth}
\usepackage{cmll}
\setlength{\parindent}{5cm}

\DeclarePairedDelimiter\ceil{\lceil}{\rceil}
\DeclarePairedDelimiter\floor{\lfloor}{\rfloor}

\newtheorem{propo}{Proposition}

\title{Polynomials and Graphs}
\date{2013-2014 \nth{1} semester}

\begin{document}
\maketitle

\newpage 
\tableofcontents

\newpage
\section{Introduction}
Existence results $\rightarrow$ computation, counting\\

\section{Polynomials}
\paragraph{Min number $C_n$ of complete bipartite graphs to cover $K_n$\\}
$C_n = ceil (\log_2 n)$\\
Existence: To each vertex give a word in $\{0,1\}^{C_n}$, each bipartite $B_i$ is between vertices of $i$th bit $= 0$\\
Minimal: exercise\\

\paragraph{Bipartite graphs partitioning $K_n$\\}
\begin{itemize}
\item At most $n-1$
\item Graham-Pollak '71: optimal\\
	Tverberg '81: Consider vertices of $K_n$ as variables $(x_i)_{1 \leqslant i \leqslant n}$\\
	Assume $E(K_n) = \bigcup\limits_{1 \leqslant i < l-1} E(K_{A_i,B_i}$ with $l < n-1$ and $A_i, B_i \subset V(K_n)$\\
	\[(\sum x_i)^2 = \sum\limits_{i=1}^n x_i^2 + 2 \sum\limits_{1 \leqslant i < j \leqslant n} x_i x_j\]
	\[ = \sum\limits_{i=1}^n x_i^2 + 2 \left( \sum\limits_{i=1}^l \left( \sum\limits_{x_j \in A_i} x_j \right) \left( \sum\limits_{x_k \in B_i} x_k \right) \right) \]
	Consider $(x_i)$ not all zero with $\forall i \leqslant l, \sum\limits_{x_j \in A_i} x_j = 0$ and $\sum\limits_{i=1}^n x_i = 0$\\
	Less than $n$ constraints: there is such an $(x_i')$\\
	Using previous lines: $\sum\limits_i x_i^2 = 0$ contradiction
\item Combinatorial argument for $\geqslant c * n$ for constant $c$ ??	
\end{itemize}

\subsection{Chromatic Number}
Let $G=(V,E)$ graph, $S \subset V$ is:
\begin{itemize}
\item stable set when $\forall x <> y \in S, (x,y) \notin E$
\item clique when $\forall x <> y \in S, (x,y) \in E$
\end{itemize}
$\alpha(G) :=$ maximum size of stable set\\
$\omega(G) :=$ maximum size of clique e.i. clique number\\
$G$ is $k$-colorable when $\exists C : V \rightarrow [[1;k]]$ such that $\forall (x,y) \in E, C(x) <> C(y)$\\
For $C$ a $k$-coloring, the set of vertices with a given color is called a color class\\
$k$-color $\Leftrightarrow \exists$ partition into $K$ stable sets\\
The chromatic number $\xi(G) :=$ min $k$ such that $G$ is $k$-colorable\\
A certificate $\xi(G) \leqslant k$ is a $k$-coloring\\
A certificate $\xi(G) \geqslant k$ is hard\\
$\omega(G)$ is a lower bound.

\paragraph{$\omega = 2$:} triangle-free graph\\
Construction: Mycielski '55\\
\begin{itemize}
\item $M_2 :=$ line
\item $M_3 :=$ pentagon
\item $M_4 :=$ copy $M_3$, then copy its points. For a point copy link it to the originals of its neighbors + a new universal point
\end{itemize}
Properties:
\begin{itemize}
\item triangle free
\item $\xi (M_i) = i$
\item Every $M_i$ is critical: if a vertex is removed then the chromatic number is less than $i$ (possibly edge critical)
\item The sequence $(M_i)_i$  is universal: $\forall G$ triangle free, $\exists i, M_i$ contains $G$ as an induced subgraph (obtained by removing vertices)
\end{itemize}

Definition: the girth of graph is the minimum length of a cycle in the graph\\
e.g. triangle free $\Leftrightarrow$ girth at least $4$\\
BFS of depth $girth/2$ from any vertex is a tree\\\\

Construction: Erdos '59\\
$\forall  g, \exists G, \xi(G) \geqslant g \wedge girth(G) \geqslant g$\\
\begin{itemize}
\item take a random graph on $n$ vertices and $p \left( (x,y) \in E \right) \frac{1}{n^{1-\varepsilon}}$ for $\varepsilon$ small\\
	Expected degree: $n^\varepsilon$\\
	$\mathbb{E}($ cycles of length $< g) = o(n)$\\
	$\mathbb{E} \left( \alpha(G) \right) = o(n)$
\item For every constant $c \in \mathbb{N}, \exists G$ with the number of cycles of length $< g$ is less than or equal to $\frac{n}{c}$
	and $\alpha(G) \leqslant \frac{n}{c}$\\
	In $G$ delete $\leqslant \frac{n}{c}$ vertices to remove cycles of length $< g$, this produces $H$\\
	\[\alpha(H) \leqslant \frac{n}{c}\]
	$girth(H) \geqslant g$ and
	\[\xi(H) \geqslant \frac{|V(H)|}{\alpha(H)} \geqslant \frac{\frac{\left( c-1 \right) .n}{c}}{\frac{n}{c}} = c-1\]
\end{itemize}

\paragraph{Kneser graphs\\}
Fix $0 < k < n$ integers\\
The kneser graph $Kn(k,n)$ has a set of vertices $\binom{2n+k}{n}$ identified to subset\\
Edges are between non intersecting sets\\
NB: still defined for $k \geqslant n$\\
If $k < n$ then $Kn(k,n)$ is triangle free\\
For $k$ fixed and $n \rightarrow \infty$, no short-odd cycles: odd girth is large $\Leftrightarrow$ every small induced graph is bipartite\\
Conjecture: $\xi \left( Kn(k,n) \right) = k+2$ (Kneser '55)\\
\begin{itemize}
\item there is a $k+2$ coloring:\\
	Color with $1$ all vertices containing $1$\\
	Color with $2$ all remaining vertices containing $1$\\
	Keep going until $k$, the remaining graph is $X \subset [[ k+1; 2n+k]]$, it is a matching: $2$ more colors\\
	Result: $k+2$-coloring
\item Lovasz '78:\\
	Assume that $\exists c$ a $k+1$-coloring.
	\begin{itemize}
	\item Distribute the dots $[[1; 2n+k]]$ on the $k$-dimension sphere $S^k$ (living in dimension $k+1$) such that in every open half sphere there are at least $n$ dots
	\item Transport $c$ on the sphere: every point $x \in S^k$ is the pole of some half-sphere $H_x$.\\
		$x$ sees at least $n$ elements. $\exists X \subset [[1; 2n+k]]$ of size $n$, $X \subset H_x$.\\
		\[c'(x) := c(X)\]
		If multiple $X$ are possible give every color to $x$ (multicoloring)\\
		$c'$ is a $k+1$-multicoloring of $S^k$\\
		Every color induces an open subset of $S^k$\\
	\item Borsuk-Ulam: every covering of $S^k$ by $k+1$ open sets has $2$ antipodal points $x$ and $\bar{x}$ with the same color.
		Pick $X$ and $bar{X}$ giving the color, $c(X) = c(\bar{X})$ but $(X,\bar{X}) \in E \left( Kn(k,n) \right)$
		CONTRADICTION: no $k+1$-coloring
	\end{itemize}
	Additionally, $\forall G, \xi(G) \geqslant$ topological connectivity of the neighborhoods of $G$
\end{itemize}

\subsection{Edge coloring}
A $k$-edge coloring of $G$ is a function $c : E \rightarrow [[1;k]]$ such that $\forall (e,f) \in E^2, e \bigcap f \rightarrow c(e) <> c(f)$\\
(where intersection means having a vertex in common)\\
Color classes are matchings, $\exists k$-edge coloring $\Leftrightarrow E(G)$ partitionnable into $k$ matchings\\
Chromatic index $\xi'(G)$ is the minimum $k$ such that $G$ is $k$-edge colorable.\\
Trivial lower bound: $\Delta(G)$ maximum degree of $G$\\
Computing $\xi'$ is NP-hard\\

\paragraph{Vizing '64:} $\forall G, \xi'(G) \leqslant \Delta(G) + 1$\\
Proof by Schrijver\\
A graph is "class 1" when its chromatic index is its degree, "class 2" otherwise.\\
Case: $\Delta$-regular graph (all vertices have degree $\Delta$)\\
Class 1 $\Leftrightarrow \exists$ $\Delta$ perfect matchings (matching covering all vertices) partitioning $E$\\
Peterson 1890: every cubic (3 regular) connected bridgeless graph has a perfect matching.\\
(Bridgeless: removing an edge leaves the graph connected)

\paragraph{Snark: cubic bridgeless connected class 2 graph\\}
Tait 1890: $4$-color theorem $\Leftrightarrow$ no snark is planer\\
Seymour cycle double cover: every bridgeless graph has a collection of cycles covering every edge twice $\Leftrightarrow$ true on snarks.\\
Tutte nowhere 5 flow conjecture $\Leftrightarrow$ true on snarks.\\

\paragraph{Line graph:} $L(G) := (E,\{(e,f) | e <> f \wedge e \bigcap f <> 0 \}$\\
Then $\xi(L(G)) = \xi'(G)$\\

\subsection{Matchings}
\paragraph{Hall 1936:} in a bipartite graph $(X,Y), \exists$ a matching saturating $X \Leftrightarrow \forall X' \subset X, |N(X')| \geqslant |X'|$.\\
A contracting set $X'$ ($|N(X')| < |X'|$) is a co-NP certificate: there is no matching covering $X$.\\
Equivalently in a hypergraph $H=(V,E)$ where $E \subset P(V), \exists$ an independant set of representative (ISR) (choice of pairwise distinct elements in $e \in E$) $\Leftrightarrow \forall E' \subset E, |\bigcup E'| \geqslant |E'|$\\
Transform hypergraph into graph: for $e \in E, v \in V, (e,v)$ is an edge when $v \in e$.\\

\paragraph{Tutte 1947 and Berge 1958:\\}
A graph $G=(V,E)$ has a perfect matching $\Leftrightarrow \forall S \subset V,$ the number of odd connected components of $G \backslash S$ is at most $|S|$.\\
(odd component: all vertices are connected and not connected to the rest of the graph)\\
This gives a certificate that there is no perfect matching.\\
Perfect matching in graphs is NP $\bigcap$ co-NP.\\
Edmonds $1965$: polynomial algorithm.

\paragraph{Corollary:} every bridgeless cubic graph has a perfect matching.\\
Let $S \subset V$, show that $|odd(G \backslash S)| \leqslant |S|$.\\
Consider $C = (c_1 ... c_l)$ odd components. The number of edges leaving $c_i$ is greater than $2$ (graph is brigdeless).\\
The number of edges leaving $c_i$ is $\sum\limits_{x \in c_i} d(x) = 2 e(c_i) + e(c_i, S)$ and $= 3 |c_i|$.\\
The number of edges leaving $\bigcup c_i$ is $\geqslant 3.l$\\
The number of edges entering $S$ is $\leqslant 3.|S|$\\
\[|S| \geqslant l\]

\paragraph{Konig 1916:} Bipartite graphs are class 1.\\
Let $G=(X,Y)$ bipartite.\\
$\exists$ a matching covering all vertices with degree $\Delta$. Iteratively remove this matching: this gives a partition into $\Delta$ matchings.\\
\begin{itemize}
\item Proof 1: $\exists$ a matching $M_X$ covering all $\Delta$ vertices in $X$.\\
	It comes from Hall: $\forall X' \subset X_\Delta$ the number of edges leaving $X'$ is $\Delta.|X'|$.\\
	The number of edges received is at most $\Delta.|N(X')|$, then $|N(X')| \geqslant |X'|$.\\
	Similarly $\exists M_Y$ covering $\Delta$ vertices in $Y$.\\
	Orient $M_X$ from $X$ to $Y$ and $M_Y$ from $Y$ to $X$.\\
	Extract a matching covering vertices of positive out-degree. Non covered vertices have out-degree $0$ so are not in $X_\Delta$ or $Y_\Delta$.\\
\item Proof 2: using polytopes.\\
	Consider a matrix such that the lines are edges and the columns are matchings.\\
	A cell is $O$ if the edge belongs to the matching, $1$ otherwise.\\
	NB: one column is zeroed.\\
	Vectors are points in $\mathbb{R}^m$ with $m$ number of edges.\\
	The convex hull is the Bipartite Matching Polytope $MP(G)$.\\
	An element is a point with coordinates $x_{i,j}$ function of the edge $(i,j)$.\\
	\emph{Theorem:} $MP(G)$ is described by the following constraints:
	\begin{itemize}
	\item $x_{i,j} \geqslant 0$
	\item $\forall i in X, \sum\limits_{i,j \in E}x_{i,j} \leqslant 1$
	\item $\forall i in Y, \sum\limits_{i,j \in E}x_{i,j} \leqslant 1$
	\end{itemize}
	The point $\frac{1}{\Delta} . \mathbb{1}$ belongs to the polytope and realises inequalities with equality for degree $\Delta$ vertices.\\
	Therefore it is realised as a barycenter of matchings $\sum\limits_{i \leqslant l} \varepsilon_i M_i$\\
	Each $M_i$ must realise equality on $\Delta$ vertices: it covers degree $\Delta$ vertices.\\
\item Proof 3: by matroids.\\
	Matroid: hypergraph $H=(S,I)$ ($S$ ground set, $I$ hyperedges i.e. independent) with conditions:
	\begin{itemize}
	\item $I <> 0$
	\item $\forall f \in I, \forall e \subset f, e \in I$
	\item $\forall e, \forall f \in I, |e| < |f| \rightarrow \exists x \in f \backslash e, e \bigcup \{x\} \in I$
	\end{itemize}
	e.g. for $S$ set of vectors, $e \in I \Leftrightarrow$ e is a set of linearly independent vectors\\
	A base is a largest size edge.\\
	The rank of the matroid is the size of a base.\\
	For $(V,E)$ graph, $(E, \{F \subset E | F$ acyclic$\})$ is the graphic matroid.\\
	If the graph is connected bases are spanning trees.\\
	When elements of $S$ have weights, the greedy algorithm produces a maximum weight basis.\\
	\emph{Parition matroid:} if $S$ has a partition into $S_1 ... S_l$ let $I := \{ S' \subset S, \forall i, |S' \bigcap S_i| \leqslant 1 \}$\\\\
	%%%%%%% left empty deliberately %%%%%%%%%%%
	Let $M_1=(S,I_1)$ and $M_2=(S,I_2)$ matroids. $M_1 \bigcap M_2 := (S,I_1 \bigcap I_2)$ where we keep hyperedges which are independents in both matroids.\\
	Bipartite matchings are matroid intersections: let $(X,Y)$ bipartite with edges $E$.\\
	$H := (E,M)$ with $M := \{E' \subset E$ matching$\}$.\\
	$M_1$ is the set of edges with degree in $X$ at most 1.\\
	It is a partition matroid.\\
	$M_2$ same for $Y$.\\
	Then $M_1 \bigcap M_2 = H$.\\
	\emph{Theorem} (Edmonds): computing a largest hyperedge in $M_1 \bigcap M_2$ is in P.\\
	Also rank$(M_1 \bigcap M_2) = \min\limits_{X \subset S} (rank_{M_1}(X) + rank_{M_2}(S \backslash X)$\\
\end{itemize}

\subsection{List coloring}
Make coloring harder to capture more problems and have more tools.\\
Erdos-Rubin-Taylor: the input of a list coloring problem is a graph $G=(V,E)$ and a function $L$ such that $\forall v \in V, L(v)$ is a list of colors (integers).\\
The output is a coloring $c : V \rightarrow \mathbb{N}$ with
\begin{itemize}
\item $\forall v, c(v) \in L(v)$
\item $\forall (u, v) \in E, c(u) <> c(v)$
\end{itemize}
Also called an $L$-coloring.\\

\paragraph{Transform to simple coloring:} give the same list to all vertices.\\

\paragraph{List chromatic number $\xi_l(G)$} is the minimum $k$ such that when all lists have size $k$ there exists a list coloring.\\
$\xi_l(G)$ can be greater than $2$ for bipartite graphs.\\
We can have $\xi_l(G) > \xi(G)$.\\
$\exists$ bipartite graph with arbitrarily large $\xi_l$.\\
$\forall G, G$ planar $\rightarrow \xi_l(G) \leqslant 5$ (proof by C. Thomassen).\\

\paragraph{Brooks's theorem:} if $G$ connected is not a clique nor an odd cycle then $\xi \leqslant \Delta$\\
\begin{itemize}
\item If $\exists x \in V, d(x) < \Delta$ then $G$ is $\Delta$-colorable.\\
	Start with a search tree from $x$. Consider an elimination order.\\
	$\forall i > 1, x_i$ has a neighbor with index less than $i$.\\
	Color greedily: $x_i$ has at most $\Delta-1$ colors used in its neighborhood when it is colored. $x_1$ has degree $< \Delta$.\\
\item Now consider $G$ is $\Delta$-regular. We can assume $G$ is 2-connected (no cut vertex) (will not work for lists).\\
\item Pick $x$ a make a depth first search tree: spanning tree such that branches are only linked through comon ancestors.\\
	If not a path: pick a branching point $z$ with children $u$ and $v$.\\
	If $u$ and $v$ are removed the graph remains connected (because DFS and item 2).\\
	Make an elimination order $(z \equiv x_1) \; x_2 ... x_{n-2}  \; u \; v$. Color $u$ 1, $v$ 1. $x_1$ will be colorable.\\
	Use greedy algorithm, finish.
\item If $\forall x \in V, \forall$ DFS rooted at $x$ we get an hamilton path.
	If $G$ is not a clique, $\exists a \; b \; c, (a,b) \in E \wedge (b,c) \in E \wedge (a,c) \notin E$\\
	If $\Delta > 2$ the case of cycles is easy.\\
	Root a DFS from $a$ starting $a \; b \; c$. It is a hamilton path.\\
	Removing $a \; c$ does not disconnect: back to previous item.
\end{itemize}
This proof can be adapted to lists (Brooks) (consider 2-connected first).

\paragraph{Erdos, Rubin, Taylor\\}
If $\forall x \in V, |L(x)| = d(x) \rightarrow G$ is $L$-colorable.

\subsection{Dinitz problem}
Let $M$ a $n \times n$-array where $\forall i \; j, m_{i,j}$ is a set of size $n$ (set of integers).\\
Can we choose 1 element per $m_{i,j}$ such that the result is a latin square (lines/columns have pairwise distinct elements)?\\
\[ \exists? f : \forall i \; j, m_{i,j} | \forall i \; j \; j', j<>j' \rightarrow m_{i,j} <> m_{i,j'} \wedge m_{j,i} <> m_{j',i} \]
Galvin: kernel method.\\
Alon-Tarski: polynomial method.\\
Graph representation: vertices are in $n \times n$, edge when distance 1. $L(x) = m_x$, $\xi_l(G_{n,n}) ?= n$\\
Erdos Rubin Taylor: $\Delta = 2n-2$. We divide by 2 by orienting edges.\\
$G_{n,n}$ is line graph of $K_{n,n}$\\
Generalise: $\xi_l'(G) = \Delta$ when bipartite
\[ \forall G, \xi_l'(G) ?= \xi'(G) \]
Note that $\xi_l'(G) \leqslant 2\Delta - 2$.\\

\paragraph{Jeff Kahn:} $\xi_l'(G) \leqslant \Delta + o(\Delta)$\\
Semirandom methods + hardcore distribution on the matching polytope\\
"draw a random matching in a graph"\\

\paragraph{Galvin proof:} kernel method\\
A kernel in a directed graph $D=(V,A)$ is $S \subset V$ such that
\begin{itemize}
\item $S$ is a stable set
\item $S$ is dominated: $\forall x \notin S \exists y \in S, (x,y) \in A$
\end{itemize}
From game theory: winning strategy\\
Odd induced cycles have no kernels\\
Where to find kernels:
\begin{itemize}
\item bidirected graphs: $\forall (x,y) \in A, (y,x) \in A$\\
	Use a stable set maximum for inclusion.
\item Acyclic directed graphs\\
	Start with $S=\emptyset$. While $D <> \emptyset$,\\
	\[S := S \bigcup sink(D)\]
	\[D := D \backslash \{ sink(D) \bigcup N^- (sink(D)) \} \]
	Finally return $S$.
\item bipartite graphs
\end{itemize}
Common property: every induced subgraph of $D$ has a kernel.\\

\paragraph{Bondy, Boppana, Siegel:} if $D$ is kernel perfect and every vertex $x \in V$ has a list $L(x)$ of size $d^+(x) + 1$ then $D$ is $L$-colorable.\\
Fix a color $1$ appearing in some list, $V_1 = \{ x, 1 \in L(x) \}$\\
$D_1$ the graph induced by $V_1$, it has a kernel $K_1$.\\
Color $1$ all elements of $K_1$ (stable set) and delete $1$ from the list of vertices of $D \backslash K_1$\\
$1$ less color and every edge not in $K_1$ has out degree decreased.

\paragraph{Dinitz proof}
\[ \xi_l = 4 = d^+(x) + 1 \]
We need an orientation where $\forall x, d^+(x) \leqslant 3$ and kernel perfect.\\
In particular triangle free.\\
Lines and columns are cliques, orientation is total order.\\

\paragraph{Sands, Samer, Woodrow 1982\\}
If a directed graph is the union of 2 (strict) partial orders on the same underlying set, then it is kernel perfect.\\
Seho's proof: induction on $|V|$, note $P_1$ and $P_2$ the orders.\\
Let $M_1$ the set of maximum elements in $P_1$. If stable set it is a kernel, otherwise $\exists (x,y)$ edge in $P_2$ with $x$ and $y$ in $M_1$.\\
Remove $x$, apply induction on $D \backslash x$ to get a kernel $K$ on it.\\
If $y \in K, K$ is a kernel for $D$.\\
If $y \notin K, \exists z \in K, y \rightarrow z$ but $y \in M_1$ so $y \rightarrow^{P_2} z$ and $x \rightarrow z$. Then $K$ is a kernel.

\paragraph{Consequences}
\begin{itemize}
\item bipartite graphs are kernel perfect (up edges + down edges)
\item Let $T$ an orientation for a complete graph (a tournament). Color the edges in 2 colors arbitrarily.\\
	$\exists u \in V, \forall v \in V, \exists$ monocolor path between $u$ and $v$.\\
	Consider the transitive closure of [color] arcs: 2 partial orders. $T$ is included in their union which has a kernel.\\
	The kernel is a vertex (no stable sets size 2 in tournament), name it $v$.\\
	This vertex is dominated for any vertex by the [color] transitive closure: exists a [color] path.\\
	Open: color with 3 colors, $? \exists S \subset V, |S|$ bounded such that $\forall x \notin S \exists$ monochromatic path from $S$ to $x$.\\
	Bound expected to be $3$.\\
\item Every triangle free orientation of the line graph of $K_{n,n}$ is kernel perfect ($\Rightarrow$ Dinitz).\\
	Vertical + horizontal partial orders
\item \emph{Stable Marriage Theorem}\\
	Sets $B$ and $G$ of size $n$.\\
	Each element of $B$ (resp $G$) is associated with a total order on $G$ (resp $B$): triangle free orientation of $L(K_{n,n})$.\\
	$\exists$ kernel in the line graph: set of edges. Stable: perfect matching, dominated: no edge such that it is preferred to the chosen one by both sides.
\item Galvin: for every bipartite graph, $\xi_l'(G) = \Delta$\\
	Orient the line graph $L(G)$ such that it is triangle free and $\Delta^+ \leqslant \Delta - 1$\\
	The edges can be $\Delta$-colored (because $G$ class 1).\\
	Orientation: $\forall e \; f$ edges in $G$, $\{e,f\} \in L$ when incident on $V_1$ xor $V_2$.\\
	Then we orient $(e,f) \in L$ if incident on $V_1$ and $c(e) < c(f)$ or incident on $V_2$ and $c(f) < c(e)$\\
	Then oriented $L$ is kernel perfect and $\Delta$ list colorable.
\end{itemize}

\paragraph{Alan Tarsi method\\}
Let $G=(V=\{ x_1 ... x_n \},E)$ graph\\
The graph polynomial $P_G(x_1 ... x_n) := \Pi_{(x_i,x_j) \in E, i<j} (x_i - x_j)$\\
Every monomial has degree $m = |E|$.\\
A factor is obtained by selecting $x_i$ or $-x_j$ for every edge: equivalent to orienting the edge.\\
In a term $\Pi x_i^{t_i}$, $t_i$ corresponds to the outdegree of $x_i$ in the orientation.\\
The coefficient of the term is $+$ or $- 1$ depending on the parity of the number of backward arcs in the orientation.\\
Let $D$ an orientation with outdegree sequence $(t_i) =: (d^+(x_i))$\\
$ O((t_i)) = \{ D'$ with degree sequence $(t_i) \}$\\
$EO((t_i)) = \{ D' \in O((t_i)), back(D')$ even $\}$\\
$OO$ same for odd backwards.\\
In $P_G$ a monomial $c . \Pi x_i^{t_i}$ verifies $c = |EO((t_i))| - |OO((t_i))|$\\
$P_G = \sum\limits_{(t_i) valid}(|EO((t_i))| - |OO((t_i))|) \Pi_i x_i^{t_i}$\\

\paragraph{If $c : V \rightarrow \mathbb{N}$  coloring then $P_G((c(x_i)_{1 \leqslant i \leqslant n})$ is nonzero when $c$ is a proper coloring.\\}

Let $D,D' \in O(d_1 ... d_n)$. Then $D \backslash D'$ is an eulerian subgraph of $D$, i.e. $\forall x_i, d^+(x_i) = d^-(x_i)$\\
Corollary: we can go from $D$ to $D'$ by inversing the orientation of a sequence of circuits.\\
($D, D'$ differ on an eulerian subgraph $E$ which is an arc-disjoint union of circuits.)\\

\paragraph{Lemma:} Let $D \in O(d_1 ... d_n)$\\
Let $\phi_D : O(d_1 ... d_n) \rightarrow E(D)$\\
$\phi_D(D') := D \backslash D'$\\
Where $E(D)$ set of eulerian subgraphs of $D$\\
Then $\phi_D$ is a bijection. $\phi_D^{-1}(D')$ reverses the arcs of $D'$ in $D$.\\
Now $OE(D) = \{$ eulerian subgraphs of $D$ with an odd number of arcs$\}$\\
$EE$ with an even number of arcs.\\
NB: $\forall D, EE(D) <> \emptyset$: $\emptyset \in EE(D)$\\

\paragraph{Lemma:} if $D \in EO(d_1 ... d_n)$ then $\phi_D(EO(d_1 ... d_n)) = EE(D) \wedge \phi_D(OO(d_1 ... d_n)) = OE(D)$\\
If $D$ even orientation then $eo(d_i) - oo(d_i) = ee(D) - oe(D)$\\
If $D$ odd orientation then $eo(d_i) - ee(d_i) = -(ee(D) - oe(D))$\\
In both cases nonzero when $ee(D) <> oe(D) \Leftrightarrow (x_1^{d_1} ... x_n^{d_n})$ appears in $P_G$\\

\paragraph{Combinatorial Nullstellensatz (Alan)\\} 
Let $P(x_1 ... x_n) \in \mathbb{F}[x_1 ... x_n]$ where $\mathbb{F}$ field such that $d^{\circ}(P) = d_1 + ... + d_n$ and $x_1^{d_1} ... x_n^{d_n}$ has nonzero coefficient.\\
Let $S_1 ... S_n \subset \mathbb{F}$ with $\forall i, |S_i| = d_i + 1$.\\
Then \[ \exists s_1 \in S_1 ... s_n \in S_n, P(s_1 ... s_n) <> 0 \]
By induction:
\begin{itemize}
\item degree $0$: $P <> 0$ and every $S_i$ is a singleton.
\item $d^\circ > 0$\\
	We can assume $d_1 > 0$. Select $s_1$ in $S_1$.\\
	If $P$ does not vanish on $\{s_1\} * ... * S_n$ we are done.\\
	If it does vanish, divide $P$ by $(x_1 - s_1)$: $P = Q . (x_1 - s_1) + R$ with $R \in \mathbb{F}[x_2 ... x_n]$ and $deg(Q) = d-1$ with nonzero coefficient on $x_1^{d_1 - 1} x_2^{d_2} ... x_n{d_n}$.\\
	$R$ vanishes on $S_2 * ... * S_n$.\\
	Induction hypothesis on Q: $s_1' \in S_1 \backslash \{s_1\}, s_2 \in S_2 ... s_n \in S_n$ and $Q(s_1', s_2 ... s_n) <> 0$ and $R(s_2 ... s_n) = 0$\\
	$P(s_1', s_2 ... s_n) <> 0$\\
\end{itemize}
Then if $D$ is an oriented graph such that every vertex $x$ has a list $L(x)$ of $d^+(x) + 1$ colors, and if $ee(D) <> oe(D)$ then $D$ is $L$-colorable.\\
NB: $oe <> ee$ can be replaced by $D$ is Kernel-perfect.\\
Common generalisation?\\
If $D$ is acyclic $ee-oe \equiv 1-0 = 1 <> 0$: greedy coloring\\
If $D$ bipartite, no odd cycles $\rightarrow oe = 0 \wedge ee > 0$\\

\paragraph{Dinitz problem} an orientation of the line graph of $K_{n,n}$ can have $ee = oe$ (e.g. with $n = 3$)\\
Conjecture: for even $n$ any orientation with $d^+ = n-1$ statisfies $ee <> oe$.\\

\paragraph{Conjecture (Erdos):} every graph $G$ which is the union of a hamilton cycle of length $3k$ and $k$ disjoint triangles has $\xi = 3$ (and $\xi_l = 3$). (Cycle + triangle conjecture)\\
Proof (Fleischner and Stiebitz): every orientation of $G$ with $d^+ = 2$, $\forall x$ statisfies $ee(D) -oe(D) \equiv 2 [4]$ and $G$ is 3-list-colorable.\\
NB: Kernel perfection n/a.\\
Problem STABLE-TRANSVERSAL:\\
Input $G=(V,E)$ with partition $V_1 ... v_n$\\
Output: $\exists?  S$ stable set such that $\forall i, |S \bigcap V_i| = 1$\\
Case: all $V_i$ have size $3$ and $G$ is a cycle of length $3k$\\
Case: $G$ disjoint union of cliques\\
Stable-Transversal is pretty much SAT.\\
Characterisation: If $\forall I \subset [[1;k]], \eta(S(G[\bigcup_{i \in I} V_i])) \geqslant |I|$ then $G$ has a S.T.\\
Corollary: IF $\forall i, |V_i| = 3$ and $G$ is a disjoint union of cycles of lengths $<> 1 [3]$ then $G$ has a S.T.\\


\section{Polynomials}
$P_G(\lambda) :=$ \# of $\lambda$-colorins of $G$ is the chromatic polynomial.\\
$P_G(\lambda) := |\{c : V \rightarrow \{n < \lambda\}, c$ colors$\}$\\
If $G$ contains a loop: $P_G = 0$\\
If $G$ clique, $P_G = \Pi_{i=1}^n (\lambda - i + 1)$\\

\paragraph{Theorem:} $P_G$ is a polynomial.\\
Let $P$ partition of $V$ into $p$ stable sets.\\
$P$ generates $\Pi_{i=1}^p (\lambda - i + 1)$ labeled colorings.\\
\[ P_G(\lambda) = \Sigma_{P \; stable} \Pi_{i=1}^{|P|} (\lambda -i + 1) \]

\paragraph{Expansion\\}
$P_G = \lambda^n - m \lambda^{n-1} + ...$ with $n := |V|$ and $m := |E|$\\\\

Size of union: for $S$ universe and $A_1 ... A_l \subset S, |S \backslash \bigcup_i A_i| = \Sigma_{I \subset [[1,l]]}(-1)^{|I|}|\bigcap_{i \in I} A_i|$\\\\

$S = V \rightarrow [[1,l]]$ and for each $e \in E, A_e = \{ f : S, f (e_1) = f (e_2) \}$
\[ P_G = \Sigma_{I \subset [[1,m]]} (-1)^{|I|} |A_I| \]
$|A_I| = \lambda^{k(G_I)}$ where $k(G_I)$ number of connected components in $G_I$\\
\[ P_G = \Sigma_{I \subset E} (-1)^{|I|} \lambda^{k(G_I)} \]

\paragraph{Graph theory\\}
Notations: $G \backslash e$ graph obtained by removing $e$ and $G / e$ obtained by contracting $e$.\\
NB: the contracted graph may have a multiple edge. We simplify them into simple edges.\\

\[ P_G = P_{G \backslash e} - P_{G/e} \]

\paragraph{Tree case\\}
By considering the edge $e$ between a leaf $f$ and its immediate ancestor\\
$T \backslash e = T/e + \{f\}$\\
$P_G = \lambda P_{G/e} - P_{G/e}$\\
$P_G = (\lambda - 1) P_{G/e}$\\
$P_G = \lambda . (\lambda - 1)^{n-1}$\\
If $P_G$ has this value then $G$ is a tree.\\

\paragraph{Cycle\\}
$P_{C_n} = \lambda.(\lambda - 1)^{n-1} - P_{C_{n-1}}$\\
$P_{C_n} = (-1)^{n-1} \lambda . \Sigma_{i=1}^{n-1} (-1)^i (\lambda - 1)^i$\\
$P_{C_2} = \lambda . (\lambda - 1)$ correct\\\\

$P_{C_n} = (-1)^{n-1} \lambda \left[ \frac{(1-\lambda)^n - 1}{- \lambda} - 1 \right]$\\
$P_{C_n} = (-1)^{n} \left[ (1-\lambda)^n - 1 + \lambda \right]$\\
$P_{C_n} = (-1)^n (1 - \lambda) [ (1 - \lambda)^{n-1} - 1 ]$\\
$P_{C_n} = (\lambda - 1)^n + (-1)^n (\lambda - 1)$\\

\paragraph{Wheel\\}
Cycle + vertex in the middle\\
(exercise)\\

\paragraph{Interval graph\\}
Consider a family of intervals, they are the vertices and two are linked when they intersect.\\
i.e. intersection graph of a set of intervals of the real line\\
Coloring is resource assignment e.g. rooms\\
$\exists S$ set of graphs such that $G$ is interval $\Leftrightarrow G$ does not contain a member of $S$ as a subgraph (1962 Lekkerkerker Boland)\\

Intervals $[a_i, b_i]$ with $b_1 \leqslant b_2 \leqslant ... \leqslant b_n$\\
There are local cliques: if $1$ and $j$ are connected all in between are connected\\
$P_G = (\lambda - \deg [a_1, b_1]) P_{G \backslash [a_1,b_1]}$\\
$\mu_1 := |\{i, b_1 \in I_i\}|$\\
$P_G = (\lambda - \mu_1 +1) P_{G \backslash [a_1,b_1]}$\\
$P_G = \Pi_{i=1}^n (\lambda - \mu_i + 1)$\\
If $\forall i, \lambda  \geqslant \mu_i$ then $P_G(\lambda) <> 0$\\
$\max_i \mu_i = |$max clique$|$\\
Every interval graph has a $\omega$-coloring with $\omega$ size of largest clique, i.e. perfect graph\\

\paragraph{The following are equivalent:\\}
\begin{itemize}
\item $G$ graph such that every induced subgraph has a simplicial vertex (neighborhood is clique)
\item $G$ contains no cycle of length $\geqslant 4$ as an induced subgraph
\item Every inclusion-minimal cut set of $G$ is a clique, where a cut set is a set of vertices such that removing it disconnects the graph
\end{itemize}
They imply the existence of an $\omega$-coloring.\\

\paragraph{$P_G = \lambda^n + \Sigma_{i=1}^{n-1} (-1)^i a_{n-i} \lambda^{n-i}$\\}
With $\forall i, a_i \geqslant 0$\\

For $G$ tree or connected graph, $\forall i > 0, 0 < a_i$\\
Induction: tree base case, otherwise pick an edgein a cycle\\
Additionally $a_{n-1} < a_{n-2} < ... a_{\floor*{\frac{n}{2}} + 1}$ (as with binomials)\\
Recognising chromatic polynomials: open\\

\paragraph{Non connected graphs\\}
$P_G$ is the product of the chromatic polynoms of connected components\\\\

A graph is 2-connected is it has at least 3 vertices and no cut set which is a singleton (cut vertex)\\
Equivalently $\forall u <> v$ vertices, $\exists$ 2 vertex disjoint paths from $u$ to $v$.\\
A graph is non-separable connected if it has at least 2 vertices and no cut vertex.\\
A block is a maximal non-separable induced subgraph.\\
Every connected 2+ vertices graph has a block decomposition (every edge is in a unique block)\\
If the block decomposition is $B_1 ... B_l$ then $P_G = \frac{1}{\lambda^{l-1}} \Pi_i B_i$\\

\paragraph{Roots of $P_G$\\}
O is a root\\
If $G$ connected all coefficients are $<>0$ and $0$ has multiplicity 1.\\
Otherwise multiplicity of O is the number of connected components\\
If $G$ is connected with 2+ vertices then 1 is a root with multiplicity the number of blocks\\
NB: in every 2-connected graph, $\exists e$ edge such that $G/e \vee G \backslash e$ is a block\\

\paragraph{The number $R_G$ of acyclic orientations of a graph $G$ is $(-1)^n P_G(-1)$}
\[ R_G = R_{G/e} + R_{G \backslash e} \]

\paragraph{Petersen Theorem: every bridgeless cubic graph has a perfect matching\\}
\emph{Vizing: $\forall G, \xi'(G) \leqslant \Delta(G)+1$\\}
Proof: $\forall k, S_k := \forall v \in V,$ if it and its neighbors have degree at most $k$ and at most one has degree $k$,\\
then if $G \backslash v$ has a $k$-edge coloring so does $G$\\
Apply with $k = \Delta+1$\\

\paragraph{Kuratowski theorem: $G$ is planar if and only if it does not contain $K_5$ as a minor\\}
(minor $H$ of $G$: $H$ obtained from $G$ by deleting vertices and deleting edges and contracting edges)\\
\emph{Hadwiger conjecture:} if a graph has no $K_l$ minor then it has a $l-1$ coloring\\
True for $l \leqslant 4$ ($l=4$: uses series parallel graphs)\\
$l=5 \Rightarrow 4-$color theorem. Wagner: $4-color$ theorem $\Rightarrow l=5$ case.
$l=6:$ Robertson, Seymour, Thomas\\
$l \geqslant 7:$ open\\

\paragraph{Construction of graphs with no $K_5$ minor\\}
Start with planar graphs and $W$\\
Glue on ($\leqslant 3$)-cliques and erase edges\\
(where $W$ the Mobius ladder)\\

\paragraph{Mader} $\exists h$ function such that every graph of average degree $h(l)$ contains a subgraph isomorphic to a subdivision of $K_l$\\
(subdivision replaces an edge with a path)\\
Then graphs with no $K_l$ minor have bounded chromatic degree.\\

\subsection{List coloring of planar graphs\\}
Thomasen:if $G$ planar and a 5 color list assigned to each vertex, then $G$ has a coloring.\\
Restrict to near triangulation: all faces are triangles except outer face. (A planar graph can add edges to become a near triangulation)\\

\emph{If all vertices of a planar graph have even degree, then the graph is 2 face-colorable\\}

\paragraph{A 3-connected cubic planar graph is 4 face-colorable if and only if it is 3 edge-colorable\\}
Using the Klein group $(\{a,b,c,e\}, +)$ with $x+e = x$, $+$ commutative, $x+x=e$ and $\forall x \; y \; z$ distinct not $e$, $x+y=z$.\\
eg vectors $(0,1),(1,0),(1,1),(0,0)$\\

\paragraph{Not every  cubic planar graph is hamiltonian (Tutte)\\}
Grinberg's theorem: if $G$ is a planar graph with hamiltonian cycle, then
\[ \sum_{i=1}^n (i-2) (\phi_i' - \phi_i'') = 0 \]
where $phi_i'$ number of faces of degree $i$ in interior of cycle, $\phi_i''$ same with exterior\\

\subsection{Nowhere zero flow}
$G$ directed graph, $\Gamma$ abelian group. A flow $\phi : E \rightarrow \Gamma$ such that
\[ \forall v, \Sigma_{e \in \delta^+(v)} \phi(e) = \Sigma_{e \in \delta^-(v)} \phi(e) \]
and nowhere zero: $\forall e, \phi(e) <> 0$\\\\

A graph has a nowhere zero flow $\Leftrightarrow$ one of its orientations has one.\\\\

For $G$ graph and $D,D'$ orientations, $\exists$ natural bijections between flows on $D$ and $D'$.\\\\

$G$ has a $\mathbb{Z}_2$ nowhere zero flow when $G$ is even: all vertices have even degree.\\\\

A $\mathbb{Z}_2 \times \mathbb{Z}_2$ (= Klein group) NZF is equivalent to being 3 edge colorable.\\

\paragraph{Counting NZF\\}
$\forall \lambda \in \mathbb{N}, \exists \Gamma_\lambda$ group of order $\lambda$. Let $Q_G(\lambda) :=$ number of NZF on $\Gamma_\lambda$.\\
We count for a fixed orientation of $G$ but does not depend on it.\\
NB: we may have loops and multiple edges.\\
If $G$ has only loops, $Q_G(\lambda) = (\lambda-1)^m$ (where $m := |E(G)|$).\\
If there is non loop edge $e$, $Q_G(\lambda) = Q_{G/e}(\lambda) - Q_{G \backslash e}(\lambda)$.\\
Note that $Q_G(\lambda)$ is independent of $\Gamma_\lambda$ and is a polynomial in $\lambda$.\\

\paragraph{Dual of a planar graph\\}
If $G$ is a planar graph, the plane has connected components. They are the vertices of the dual and are connected for each edge between them.\\
This may give rise to multiple edges and loops.\\
\emph{Theorem:} for $G$ plane graph, $P_{G^*}(\lambda) = \lambda Q_G(\lambda)$ where $G^*$ dual of $G$.\\
If $G$ is connected then $G^{**} = G$. WARNING dual may need to consider in plane properties. SAFE when $G$ 2-connected.\\\\

A coloring of the faces of $G$ is dual to a NZF of $G$ if at every vertex $v$ and an edge $(v,\_)$ separating faces colored $c_1$ and $c_2$,\\
with $c_1$ on the left for the edge looking up, the edge is colored $c_1 - c_2$.\\
Starting from a NZF, choose a face and give it an arbitrary color. Each adjacent face has only one possible color.\\

\paragraph{A $k$-flow is a $\mathbb{Z}$-flow with values in $[[-(k-1), k-1]]$.} It is nowhere zero if 0 not used.\\
$G$ has a nowhere zero $k$-flow if and only if it has a nowhere zero $\mathbb{Z}_k$-flow.\\
$\Rightarrow$ is obvious.\\
$\Leftarrow$: consider $\phi$ NZ  $\mathbb{Z}_k$-flow. It is transformed into a $k$-flow by taking the least positive representative for each class.\\
For each vertex we want the surplus $e(v)$ o be zero. If $\Sigma |e(v)| = 0$ then qed.\\
Minimise $\Sigma |e(v)|$. If vertices with positive surplus cannot reach vertices with negative surplus, not flow.\\
Else, find such a path, reverse it and use $k-\_$ for new values. Surplus for both ends decreases in absolute value, other vertices do not change: absurd.\\

\paragraph{Tutte conjectures:}
\begin{itemize}
\item Every bridgeless planar graph has a NZ $4$-flow: generalisation of 4 color theorem.
\item Every 4 edge-connected graph has a NZ $3$-flow. Proven for $8$ and $6$ connected graphs by Thomassen.
\item Every 4 edge-connected graph has a NZ $4$-flow. Proof by Jaeger.
\end{itemize}

\paragraph{Nash-Williams theorem:\\} %Nash-Williams single person
Consider $P$ partition of $V(G)$. $G/P := (P,$ edges between the elements of the partition$)$ (with multiple edges, no loops).\\
$G$ has $k$ disjoint spanning trees if and only if $\forall P$ partition, $|E(G/P)| \geqslant k(|P|-1)$\\
If $G$ is $2k$ connected then it has $k$ disjoint spanning trees.\\

\paragraph{Lemma: $G$ has a NZ $(\mathbb{Z}_2)^k$ flow if and only if $G$ can be covered by $k$ even subgraphs.\\}

\paragraph{Jaeger: every 2-connected graph has a NZ 8-flow\\}
Seymour: NZ 6-flow\\
Conjecture: 5-flow.\\


\subsection{Tutte Polynomial}

For $G$ graph, $c(G) :=$ number of connected components.\\
$\forall S \subset E(G), c(S) :=$ number of connected components spanned by $S$.\\
$n(G) := |V(G)|$\\

\begin{itemize}
\item Whitney polynomial of $G$: $W_G(x,y) := \sum_{S \subset E(G)} x^{c(S)-c(G)} y^{|S| - n(G) + c(S)}$
\item Tutte polynomial: $T_G(x,y) := W_G(x-1,y-1)$
\end{itemize}
Check is polynomial\\\\

For $e = (v,v)$ a loop, $T_G(x,y) = \sum_{S \subset E \backslash e} (x-1)^{c(S)-c(G)} (y-1)^{|S|-n(G)+c(S)} + (x-1)^{c(S) - c(G)} (y-1)^{|S|+1-n(G)+c(S)}$\\
$T_G(x,y) = y \sum_{S \subset E \backslash e} [(x-1)^{c(S)-c(G)} (y-1)^{|S|-n(G)+c(S)}]$\\
$T_G(x,y) = y T_{G \backslash e}$\\\\

For $e$ a bridge: $T_G(x,y) = x T_{G / e}(x,y)$\\
For $e$ non bridge, non loop: $T_G(x,y) = T_{G \backslash e}(x,y) + T_{G / e}(x,y)$\\

\paragraph{For $G, G'$ disjoint polynomials.\\}
If $G$ one of disjoint union OR gluing at one vertex, $T_G = T_{G'} T_{G''}$ (also true for flow polynomial)\\
(by induction on $|E(G)|$)\\

\paragraph{$P_G(x) = (-1)^{n(G)-c(G)} x^{c(G)} T_G(1-x, 0)$\\}
$Q_G(y) = (-1)^{m(G)-n(G)+c(G)} T_G(0,1-y)$\\
$T_G(1,1) = |\{S | c(S) - c(G) = 0 \wedge |S| - n(G) + c(S)\}|$: subgraphs  containing a spanning forest and no cycle: number of spanning forests.\\
$T_G(2,1) =$ number of acyclic subgraphs\\
$T_G(1,2) =$ number of connected subgraphs on $V(G)$\\
$T_G(2,2) = 2^{|E(G)|}$\\

\paragraph{If $G$ is plane then $T_G(x,y) = T_{G^*}(y,x)$.\\}
If $G$ plane, $S$ spanning tree, $S^* := E \backslash S$ then $S^*$ spanning tree of $G^*$\\\\

$P_{C_n}(x,y) = y + x + x^2 + ... + x^{n-1}$\\
For $T$ tree, $P_T(x,y) = x^{n-1}$\\

\paragraph{Reliability polynomial\\}
$G$ connected graph, $p \in (0,1)$, keep every edge of $G$ with probability $p$ ($\Leftrightarrow$ erase with probability $q := p-1$).\\
Consider $R_G(p) := \mathbb{P}(G$ connected$)$\\
$R_G(p) = p^{n-1} q^{m-n+1} T_G(1,1/q)$\\

\paragraph{Reidemeister theorem\\}


\[ per(X) := \sum_{\sigma \in \sigma_n} \Pi_{1 \leqslant i \leqslant n} X_{i,\sigma(i)} \]
\[ det(X) := \sum_{\sigma \in \sigma_n} \varepsilon(\sigma) \Pi_{1 \leqslant i \leqslant n} X_{i,\sigma(i)} \]

\subsection{Combinatorial interpretation of the permanent}
$G = (U,V,E)$ bipartite graph with $|U| = |V| = n$\\
Adjacency matrix $A_{i,j} := u_i v_j \in E$\\
$per(A) =$ number of perfect matchings of $G$\\\\

In directed graph $D = (V,E), |V| = n$ the adjacency matrix $A_{i,j} = (i,j) \in E$\\
$per(A) =$ number of cycle covers in $D$ (union of vertex disjoint cycles covering all vertices)\\\\

General matrices: for $A_{i,j} <> 0$ put weight $A_{i,j}$ on $u_i, v_i$ in  bipartite graphs.\\
Then $per(A) =$ sum of weights of all perfect matchings, where the weight of a matching is the product of the weight of its edges.\\

\subsection{Interpretation of Determinant}
$det(A) =$ sum of signed weights of perfect matchings (sign is that of the permutation)\\

\subsection{Computing perfect matchings}
Theorem by Edmonds:\\
Let $G=(U,V,E)$ bipartite graph, $|V| = n$\\
$A_{i,j} = 0$ if $u_i v_j \notin E$ else $X_{i,j}$\\
The following are equivalent:
\begin{itemize}
\item $G$ has a perfect matching
\item $per(A) <> 0$
\item $det(A) <> 0$
\end{itemize}
Algorithm: evaluate determinant at random points.\\

\paragraph{Schwarz Zippel Lemma\\}
Let $K$ field, and $P \in K[X_1 ... X_k]$\\
Fix set $S \subset K, |S| = m$\\
Let $d = \deg P \geqslant 1$\\
Evaluate on random points: if $Y_1 ... Y_k$ drawn independently from the uniform distribution on $S$ then
\[ \mathbb{P}[P(Y_1 ... Y_k) = 0] \leqslant \frac{d}{m} \]
eg $d=n, m=2n$ then probability of error less than $\frac{1}{2}$\\
Proof: induction on $k$:\\
$k=1$: at most $d$ roots\\
$k>1$: $P(X_1 ... X_k) = \sum_{j=0}^r P_j(X_1 ... X_{k-1})X_k^j$ for some $r$\\
Then $\deg P_j \leqslant d-j$. For random $y_1...y_k$ if $P(y_1 ... y_k) = 0$, depending on $P_r(y_1 ... y_{k-1}) = 0$:\\
If $=0$, probability at most $\frac{d-r}{m}$.\\
If $<>0$, $y_k$ is a zero of $P(y_1 ... y_{k-1}, Y)$ polynomial of degree $r$: probability at most $\frac{r}{m}$.\\
Total probability at most $\frac{d-r}{m} + \frac{r}{m} = \frac{d}{m}$\\\\

Then randomized NC: fast parallel algorithm. No known deterministic NC algorithm.\\

\subsection{Polynomial identity testing}
Input: polynomial $P$ from a fixed family $F$\\
Output: $P ?= 0$\\
If $P$ represented as a symbolic determinant: symbolic determinant identity testing (open)\\
If $P$ given as an arithmetic circuit (open)\\
If $P$ sum of monomials (easy)\\

\paragraph{Hitting set\\}
For a family $F$ of polynomials in $K[X_1 ... X_n]$, a hitting set is a set $H \subset K^n$ such that\\
$\forall P \in F, P <> 0 \rightarrow \exists X \in H, P(X) <> 0$\\

\paragraph{Constructing a perfect matching\\}

\subparagraph{Bipartite graph:} test if $G=(U,V,E)$ has a perfect matching with randomized algorithm.\\
$M := \emptyset$, if no PM exit.\\
Pick an edge $e = u_i v_j$, remove it. If resulting graph has perfect matching continue on it: $G := G \backslash e$. Else $M := M+e, G := G \backslash u_i, v_j$.\\
Repeat until done.\\
Polynomial time, sequential.\\

\subparagraph{Parallel algorithm\\}
$M := \emptyset$\\
For all $e \in E$ in parallel: if $e$ belongs to a perfect matching, add it to $M$ and remove its vertices from the graph.\\

\paragraph{Isolation lemma\\}
Consider $S = \{x_1 ... x_n\}$ and weights $w_1 ... w_n$. Let $F$ the family of subsets of $S$, the weight of a subset is the sum of the weight of its elements.\\
If the $w_i$ are independently drawn from the uniform distribution on $[[1, 2n]]$ then with probability at least $\frac{1}{2}, \exists! T \in F$ of minimum weight.\\
\emph{NB:} some weights will be repeated.\\
Call $x_i$ bad if it belongs to a subset $T$ of minimum weight but not all of them. There is a unique subset of minimum weight when there is no bad $x_i$.\\
$\forall i, Pr(x_i$ bad$) \leqslant \frac{1}{2n}$\\
Then $Pr(\exists i, x_i$ bad$) \leqslant \frac{1}{2}$\\
In fact for $w_i$ drawn uniformly and any values of $w_j, j<>i, Pr(x_i$ bad$) \leqslant \frac{1}{2n}$\\
Let $w^*[\bar{x_i}] = \min_{T \in F, x_i \notin T} \sum_{x_j \in T} w_j$ and\\
$w^*[x_i] = \min_{T \in F, x_i \in T} \sum_{x_j \in T, j<>i} w_j$\\
By cases:
\begin{itemize}
\item $w^*[x_i] \geqslant w^*[\bar{x_i}]$\\
	For any choice of $w_i$, $w_i$ is not bad: the minimum weight subsets do not contain $x_i$.
\item $w^*[x_i] < w^*[\bar{x_i}]$\\
	$x_i$ is bad $\Leftrightarrow w_i = w^*[\bar{x_i}] - w^*[x_i]$\\
\end{itemize}

\paragraph{Detection of perfect matchings in bipartite graphs\\}
$A_{i,j} = X_{i,j}$ if $u_iv_j \in E$, $0$ otherwise\\
$G$ has a perfect matching when $\det A <> 0$\\
Set $X_{i,j} = 2^{w_{i,j}}$ where the $w_{i,j}$ are drawn independently from the uniform distribution on $[[1 ... 2|E|]]$.\\
$\det A = \sum_\sigma \varepsilon(\sigma) \Pi_i X_{i,\sigma(i)} = \sum_{\sigma: perfect \; matching} \varepsilon(\sigma) 2^{w(\sigma)}$\\
If $G$ has a perfect matching:\\
Isolation lemma with $S := E, F =$ set of perfect matchings.\\
With probability at least $\frac{1}{2}, \det A = \varepsilon(\sigma^*) 2^{w(\sigma^*)} + \sum_{\sigma +/- \sigma^*} \varepsilon(\sigma) 2^{w(\sigma)}$\\
Where $\sigma^*$ minimum weight perfect matching.\\
$\det A \equiv \varepsilon(\sigma^*) 2^{w(\sigma^*)} [2^{w(\sigma^*) +1}]$\\
$\det A <> 0$\\

\paragraph{Comparison with Schwarz-Zippel\\}
Values in $[[0 , 2n]]$ or in $[[0, 2^{2n^2}]]$\\

\paragraph{Parallel algorithm to construct a perfect matching\\}
Construct the minimum weight perfect matching: $2^{w(\sigma^*)}$ is the highest power of 2 dividing $\det A$\\
If $e \in \sigma^*$, removing it changes the powers of 2 dividing $\det A$.\\
Algorithm:
\begin{itemize}
\item Construct $A$ with $A_{i,j} = 0 \Rightarrow u_i v_j \notin E$, else $A_{i,j} = 2^{w_{i,j}}$ for $w_{i,j}$ in uniform ditribution over $[[1,2|E|]]$
\item Compute $w = \det A$. If $w=0$ no perfect matching. Else compute largest $p, 2^p | w$
\item For $(u_i, v_j) \in E$ in parallel, construct $A^{(ij)}$ by removing the edge, compute its determinant $d_{i,j}$ and $p_{i,j} = \max_p, p | d_{i,j}$
\item If $p_{i,j} <> p, u_i v_j \in \sigma^*$
\end{itemize}
NB: $p_{i,j} = \infty$ for $d_{i,j} = 0$\\

\subsection{General graph version}
$G=(V,E), n = |V|$\\
Tutte matrix: $A_{i,j} = X_{i,j}$ if $i,j \in E \wedge i < j$, $-X_{i,j}$ if $i,j \in E \wedge i > j$, $0$ otherwise.\\
$G$ has a perfect matching $\Leftrightarrow \det A <> 0$\\
$\det A = \sum_{\sigma} \varepsilon(\sigma) w(\sigma)$ where $\sigma$ cycle covers in the directed graph where $\{i,j\} \in E \rightarrow (i,j) \wedge (j, i) \in E'$\\
$w(\sigma)$ product of the edge weights, $\varepsilon(\sigma) = (-1)^{|\sigma|+1}$ where $|\sigma|$ length of the cycle.\\
Cover by length 2 cycles creates non cancelled monomial: $\det A <> 0$\\
For $X_{i,j} = i,j \in M$ perfect matching, $\det A = 1$\\
If no perfect matching, a cycle cover must contain an odd cycle $\gamma \in \sigma$, choose the one with smallest vertex id.\\
Reorient to get new vertex cover $I(\sigma)$ with same $\varepsilon$. The weight $w(\gamma') = -w(\gamma)$: the monomials cancel.\\

\paragraph{Alternate matrix\\}
$A_{i,j}' = \varepsilon_{i,j} X_{i,j}$ for $i,j \in E$, $\varepsilon_{i,j} = - \varepsilon_{j,i} \in \{-1,1\}$\\

\subsection{Counting perfect matchings in planar graphs}
FKT algorithm (Fisher Kasteleyn Temperley, in statistical physics, 1960s)\\
Let $G$ planar graph, there is an orientation $\overrightarrow{G}$ of $G$ such that if we set\\
$A_{i,j} = 1$ if $i,j \in \overrightarrow{E}$, $-1$ if $i,j \in \overrightarrow{E}$, $0$ otherwise\\
Then the number of perfect matchings of $G$ is $\sqrt{\det A}$\\
Such orientations are called Pfaffian orientations.\\

\paragraph{Let $M,M'$ perfect matchings of $G$ (general graph).} $M \bigcup M'$ is a vertex-disjoint union of edges and even cycles.\\
Let $C$ even cycle of $G$ and $\overrightarrow{G}$ an orientation of $G$. We pick an orientation of $C$.\\
$C$ is oddly oriented if the number of edges with the same orientation in $C$ and $\overrightarrow{G}$ is odd (co-oriented edges).\\
NB: independent of the orientation on $C$.\\
An orientation is Pfaffian if for any pair of perfect matchings, the cycles in their union are oddly oriented.\\

\paragraph{Kasteleyn's theorem:} for every Pfaffian orientation of $G$, the number of perfect matchings is the square root of the determinant of its Tutte matrix.\\
Combinatorial interpretation of  $\det A[\overrightarrow{G}]$:\\
$\overleftrightarrow{G}$ where edges $G$ are replaced by edges in both directions, with $+/- 1$ depending on the orientation in $\overrightarrow{G}$.\\
$\det A[\rightarrow^{G}] = \sum_{\sigma \; cycle \; cover \; of \; \leftrightarrow^{G}} \varepsilon(\sigma) w(\sigma)$\\
There is a bijection between pairs $M,M'$ of perfect matchings and even cycle covers of $\leftrightarrow^{G}$ (NB: all cycles are even).\\
Define $\phi(M,M')$: split the union $M \bigcup M'$ between edges and even cycles, replace lone edges by the 2-cycle, even cycles by themselves (orienting independent of the matchings).\\
Exemple orientation for a cycle in $M \bigcup M'$: index vertices of $G$, start at least vertex then orient the incident edge from $M'$ towards it.\\\\

Then $\det A =$ number of pairs of perfect matchings $=$ square of number of perfect matchings.\\
$\det A = \sum_\sigma \varepsilon(\sigma) w(\sigma) = \sum_{\sigma \; even \; cover} \varepsilon(\sigma) w(\sigma)$ and $\varepsilon(\sigma) w(\sigma) = 1$ for even covers.\\

\paragraph{Any planar graph has a Pfaffian orientation.\\}
NB: any a tree, all orientations are Pfaffian.\\
Lemma: let $G$ connected plane graph. Fix an orientation $\tilde{G}$ of $G$, and assume all faces have an odd number of edges oriented clockwise (except maybe infinite face).\\
Then for any (simple) cycle $C$ of $G$ the number of edges of $C$ with clockwise orientation and the number of vertices of $G$ inside $C$ are of different parities.\\
And $\tilde{G}$ is Pfaffian.\\
Induction on edges for connected graph:\\
- $m=n-1$: tree: trivial\\
- pick an edge separating a finite face from the infinite face, $G' = G \backslash e$. $\tilde{G}$ orientation from induction, adding $e$ creates a finite face, orient $e$ so that that face has an odd number of clockwise edges.\\

\paragraph{Open problem: check existence of Pfaffian orientation in general graphs efficiently.\\}
Works for bipartite graphs.\\

\subsection{Coding complexity}

$\#-P$ complete: count the number of satisfying assignment for 3 SAT\\
Count perfect matchings in a bipartite graph\\

\paragraph{Toda's Theorem\\}
All problems in the polynomial hierarchy can be solved in polynomial time given an oracle for a $\#P$ complete problem.\\

\paragraph{Definition\\}
Fix alphabet $\Sigma$. Let $f : \Sigma^* \rightarrow \mathbb{N}$.\\
$f \in \#P := \exists A \in P, p$ polynomial, $\forall n, \forall x \in \Sigma^n, f(x) = |\lbrace y \in \Sigma^{\leq p(n)}, (x,y) \in A \rbrace|$\\
$\Leftrightarrow$ number of accepting paths of polynomial time non deterministic machine.\\

\begin{propo}[$FP \subseteq \#P$ where $FP$ functions computable in polynomial time]
The problem $\{ (x,y) | \mathrm{eval}(y) \leq f(x) \}$ is P.
\end{propo}

\paragraph{Definition:} $f \in \#P$ is $\#P$ complete when all functions in $\#$ polynomial time reduce to $f$.\\
Parsimonious reduction: $g(x) = f(i(x))$ with $i$ computable in polynomial time.\\
Many one reduction: $g(x) = o(f(i(x))$ with $i,o$ computable in polynomial time.\\
Oracle reduction: $g$ computable in polynomial time by a turing machine with access to an oracle for $f$.\\\\

$\#3SAT$ is $\#P$ complete for parsimonious reduction.\\
$\#$ perfect matchings is complete for many one reduction. If complete for parsimonious reduction then $P=NP$.\\
$\#$ exact 3 cover is $\#P$ complete for parsimonious reduction.\\
(Exact 3 cover: NP complete problem, in: $X, T \subseteq \binom{X}{3}$; out: number of exact covers of $X$ by elements of $T$)\\
Lemma: $\#$ exact 3 covers $\subseteq \#w$ bipartite match\\



\end{document}
